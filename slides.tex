\documentclass{beamer}
\usepackage[utf8]{inputenc}

\usetheme{Madrid}
\usecolortheme{beaver}

%------------------------------------------------------------
%This block of code defines the information to appear in the
%Title page
\title[Computational Proof Assistants] %optional
{Computational Proof Assistants}

% \subtitle{A short story}

\author[Liu, Kevin] % (optional)
{Kevin Liu}

\institute[UBC] % (optional)

\date[CUMC 2024] % (optional)
{Canadian Undergraduate Mathematics Conference, July 2024}


%End of title page configuration block
%------------------------------------------------------------



%------------------------------------------------------------
%The next block of commands puts the table of contents at the 
%beginning of each section and highlights the current section:

% \AtBeginSection[]
% {
%   \begin{frame}
%     \frametitle{Table of Contents}
%     \tableofcontents[currentsection]
%   \end{frame}
% }
%------------------------------------------------------------


\begin{document}

%The next statement creates the title page.
\frame{\titlepage}


%---------------------------------------------------------
%This block of code is for the table of contents after
%the title page
\begin{frame}
\frametitle{Table of Contents}
\tableofcontents
\end{frame}
%---------------------------------------------------------


\section{Introduction}

%---------------------------------------------------------
%Changing visivility of the text
\begin{frame}
\frametitle{Motivation}
% This is a text in second frame. For the sake of showing an example.

\begin{itemize}
    \item Writing correct software is hard!
    \item Theorem provers can ensure mathematical correctness of code (Compcert)
    \item Formal verification is becoming widely used in industry for critical tasks (Microsoft, Intel)
    \item Can formally prove many results in mathematics (4-color theorem)
\end{itemize}
\end{frame}

%---------------------------------------------------------


%---------------------------------------------------------
%Example of the \pause command
\begin{frame}
\frametitle{Coq}
\begin{itemize}
    \item Coq is an interactive proof assistant for formal verification
    \item Developed in 1984 by INRIA (France)
    \item Includes a programming language (Gallina) and can check proofs for correctness
    \item Can ``extract" Coq proofs into OCaml or Haskell scripts
\end{itemize}
\end{frame}
%---------------------------------------------------------

\section{Example Coq Programs}

\begin{frame}
\frametitle{How Coq Works}
    % Tactics and Goals
\end{frame}

\begin{frame}
\frametitle{Cheatsheet}
    % Reminder
\end{frame}
%---------------------------------------------------------
%Highlighting text
% \begin{frame}
% \frametitle{Sample frame title}

% In this slide, some important text will be
% \alert{highlighted} because it's important.
% Please, don't abuse it.

% \begin{block}{Remark}
% Sample text
% \end{block}

% \begin{alertblock}{Important theorem}
% Sample text in red box
% \end{alertblock}

% \begin{examples}
% Sample text in green box. The title of the block is ``Examples".
% \end{examples}
% \begin{proof}
    
% \end{proof}
% \end{frame}
%---------------------------------------------------------


%---------------------------------------------------------
%Two columns
% \begin{frame}
% \frametitle{Two-column slide}

% \begin{columns}

% \column{0.5\textwidth}
% This is a text in first column.
% $$E=mc^2$$
% \begin{itemize}
% \item First item
% \item Second item
% \end{itemize}

% \column{0.5\textwidth}
% This text will be in the second column
% and on a second tought this is a nice looking
% layout in some cases.
% \end{columns}
% \end{frame}
%---------------------------------------------------------

\section{The Curry-Howard Isomporhism}
\begin{frame}
\frametitle{The Curry-Howard Isomorphism}
\begin{theorem}
    % Mathematica
\end{theorem}

\begin{examples}
    \begin{itemize}
        \item $$A \implies B \equiv f: A \to B$$
        %  f is the evidence transformer, A is evidence
        \item $$A \wedge B \equiv (A, B)$$
    \end{itemize}
\end{examples}

\begin{block}{Remark}
    \begin{itemize}
    \item Direct link between computation and logic
    \item Programs (functions) are equivalent to Proofs
    \item Proofs can be run!
    \end{itemize}
\end{block}
\end{frame}

\section{References}
\begin{frame}
\frametitle{References}
    \begin{itemize}
        \item \href{https://coq.inria.fr/}{Coq Website}
        \item \href{https://github.com/coq/coq}{Coq GitHub}
        \item \href{https://softwarefoundations.cis.upenn.edu/lf-current/toc.html}{Software Foundations}
        \item \href{https://cel.hal.science/inria-00001173v6/document}{Coq in a Hurry}
        \item \href{https://www.rocq.inria.fr/semdoc/Presentations/20150217_PierreMariePedrot.pdf}{Curry Howard for Dummies}
        \item \href{https://cs3110.github.io/textbook/chapters/adv/curry-howard.html}{CS 3110 Textbook @ Cornell}
    \end{itemize}
\end{frame}

\end{document}
